  
\documentclass[a4paper]{article}
\usepackage[utf8]{inputenc}
\usepackage[ngerman]{babel}
%\usepackage[a4paper, left=2.5cm, right=2.5cm, top=2cm, bottom=2cm]{geometry}
\usepackage{amsmath}
\usepackage{graphicx}
\usepackage{multicol}
\usepackage[hidelinks]{hyperref}
\title{\includegraphics[scale=1]{img/logo}\\ Dokumentation}
%\author{ Andreas Kuk - 1525911}
\date{\normalsize \today}
\begin{document}
\maketitle
\newpage
\section*{Vorwort}

Diese Dokumentation dient dazu die Wordpress Webseite von dem Verein "Herzselbsthilfe Wiener Neustadt" zu erstellen. Es wurde wert darauf gelegt die Schritte so einfach wie möglich zu beschreiben. Der nötige Code wird im Anhang beigelegt und kann mittels Copy \& Paste hinzugefügt werden.

\newpage
\tableofcontents
\newpage
\section{Wordpress Installation}
Um Wordpress zu installieren gibt es mehrere Wege. Der einfachste Weg ist es mit der World4you integrierten Methode zu tun. Falls jedoch aus irgendwelchen Gründen einmal der Webhoster gewechselt werden sollte kann die Installation auch manuell durch geführt werden. Alle Informationen zum Download und der manuellen Installation findet man \href{https://de.wordpress.org/download/}{hier}.
\subsection{Easy Install}

Easy Install ist eine Funktion des Webhosters World4You und kann mit jedem Hosting Paket verwendet werden. Es ermöglicht die einfache Installation von Serveranwendungen wie z.B. Wordpress oder auch phpBB. Um eine Wordpress Installation mit Easy.Install durchzuführen klickt man im World4You Dashboard auf 'Easy.Install' und dann auf 'Installieren'. Als nächstes klickt man auf 'Wordpress'. Im folgendem Fenster wählt man die Option 'neue Datenbank anlegen' aus und bennent das Zielverzeichnis. Danach klickt man auf 'Installation starten'. Auf der nächsten Seite findet man nun die Logindaten für die Wordpress Seite. Die Webseite ist nun über den Domainnamen + dem Zielverzeichnis abrufbar. 

\subsection{Login}

Nach dem Abrufen der Logindaten ist es möglich die Wordpress Administrator Oberfläche zu erreichen. Um zur Login-Oberfläche zu gelangen schreibt nach den nachfolgenden Präfix zur Webseiten URL dazu: '/wp-login.php'
\\
Nun sollte eine Login Oberfläche mit folgenden Eingabefeldern angezeigt werden: 
\begin{itemize}
	\item Benutzername oder E-Mail
	\item Passwort
	\item Captcha
\end{itemize}
Nach Eingabe der Login-Daten wird man zum Wordpress Dashboard weitergeleitet.

\end{document}